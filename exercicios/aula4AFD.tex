\documentclass[12pt]{article}
\usepackage[utf8]{inputenc}
\usepackage[brazil]{babel}
\usepackage{amsmath, amssymb}
\usepackage{geometry}
\usepackage{booktabs}
\usepackage{listings}
\usepackage{tikz}
\usetikzlibrary{automata, positioning}
\geometry{a4paper, margin=2.5cm}

\title{Exercícios sobre Autômato Finito Determinístico (AFD) propostos pelo Prof. José Rui na aula LFA04}
\author{Giovanni Della Déa}
\date{}

\begin{document}

\maketitle

\section*{Definição do Autômato Finito Determinístico (AFD)}

\begin{itemize}
    \item \textbf{Estados:} $\{q_0, q_1, q_2, q_f\}$
    \item \textbf{Alfabeto:} $\{a, b\}$
    \item \textbf{Estado inicial:} $q_0$
    \item \textbf{Estado(s) de aceitação:} $q_f$
\end{itemize}

\subsection*{Diagrama do AFD}

\begin{center}
\begin{tikzpicture}[>=stealth, node distance=2.5cm]

  % Estados
  \node[draw, circle, initial] (q0) {$q_0$};
  \node[draw, circle] (q1) [below left=of q0] {$q_1$};
  \node[draw, circle] (q2) [below right=of q0] {$q_2$};
  \node[draw, double, circle] (qf) [below right=of q1] {$q_f$};

  % Transições
  \draw[->] (q0) -- node[left] {a} (q1);
  \draw[->] (q0) -- node[right] {b} (q2);
  \draw[->] (q1) edge[loop left] node {b} (q1);
  \draw[->] (q1) -- node[left] {a} (qf);
  \draw[->] (q2) -- node[right] {a} (q1);
  \draw[->] (q2) -- node[right] {b} (qf);
  \draw[->] (qf) edge[loop below] node {a,b} (qf);

\end{tikzpicture}
\end{center}
\subsection*{Tabela de Transição}

\begin{center}
\begin{tabular}{ccc}
\toprule
\textbf{Estado} & \textbf{a} & \textbf{b} \\
\midrule
$q_0$ & $q_1$ & $q_2$ \\
$q_1$ & $q_f$ & $q_1$ \\
$q_2$ & $q_1$ & $q_f$ \\
$q_f$ & $q_f$ & $q_f$ \\
\bottomrule
\end{tabular}
\end{center}

\section*{Simulações de Cadeias}

\subsection*{1. Cadeia: \texttt{ababaaab}}

\begin{lstlisting}
q0 --a--> q1
q1 --b--> q1
q1 --a--> qf
qf --b--> qf
qf --a--> qf
qf --a--> qf
qf --a--> qf
qf --b--> qf
\end{lstlisting}

\textbf{Resultado:} Aceita

\subsection*{2. Cadeia: \texttt{bababab}}

\begin{lstlisting}
q0 --b--> q2
q2 --a--> q1
q1 --b--> q1
q1 --a--> qf
qf --b--> qf
qf --a--> qf
qf --b--> qf
\end{lstlisting}

\textbf{Resultado:} Aceita 

\subsection*{3. Cadeia: \texttt{bbbbbbb}}

\begin{lstlisting}
q0 --b--> q2
q2 --b--> qf
qf --b--> qf
qf --b--> qf
qf --b--> qf
qf --b--> qf
qf --b--> qf
\end{lstlisting}

\textbf{Resultado:} Aceita 

\section*{Conclusão}

Todas as cadeias analisadas foram aceitas pelo autômato, pois terminam no estado de aceitação $q_f$. O AFD é capaz de permanecer em $q_f$ indefinidamente, aceitando qualquer continuação com símbolos válidos do alfabeto $\{a, b\}$.

\end{document}