\documentclass[12pt]{article}
\usepackage[utf8]{inputenc}
\usepackage[brazil]{babel}
\usepackage{amsmath}
\usepackage{geometry}
\usepackage{verbatim}
\geometry{a4paper, margin=2.5cm}

\title{Anotações — Sintaxe e Semântica}
\author{Disciplina: Linguagens Formais e Autômatos\\Professores: José Rui e Thiago Augusto de Oliveira}
\date{26 de fevereiro de 2025}

\begin{document}

\maketitle

\section*{O que é uma linguagem formal?}
\begin{itemize}
    \item Desenvolvida na década de 1950.
    \item Estuda e desenvolve teorias relacionadas à linguagem natural.
    \item Inclinou-se para as linguagens artificiais como C, Pascal, etc.
    \item Engloba as linguagens ordinárias da ciência da computação.
\end{itemize}

\section*{Qual o enfoque maior?}
O maior enfoque foi em aplicações de análise léxica e análise sintática.

\section*{Exemplos de linguagens formais}
\begin{itemize}
    \item Linguagens de programação
    \item Linguagens naturais
    \item Expressões matemáticas
    \item Circuitos digitais
\end{itemize}

\section*{Diferença entre sintaxe e semântica}

A \textbf{sintaxe} é a parte da gramática que estuda a disposição das palavras na frase e das frases no discurso, bem como a \textbf{relação lógica} entre elas.

\subsection*{Exemplos sintaticamente corretos:}
\begin{itemize}
    \item José bebeu água.
    \item Maria acabou a prova.
    \item As flores são belas.
\end{itemize}

\begin{verbatim}
if (a < 10) then
A = 45;
\end{verbatim}

\subsection*{Exemplos sintaticamente incorretos:}
\begin{itemize}
    \item Água bebeu José.
    \item A prova acabou Maria.
    \item As flores é bela (erro de concordância).
\end{itemize}

\begin{verbatim}
then (a < 10) if
45 = a;
\end{verbatim}

\section*{Erros semânticos}

Um erro semântico pode alterar completamente o sentido da frase.  
A ordem de análise é:

\[
\text{Análise léxica} \rightarrow \text{Análise sintática} \rightarrow \text{Análise semântica}
\]

\textbf{Exemplos:}
\begin{itemize}
    \item Eu caminho todos os dias. \hfill (\textit{caminho = ato de andar})
    \item O caminho é longo. \hfill (\textit{caminho = estrada})
    \item Vou colher flores. \hfill (\textit{colher = verbo})
    \item A colher caiu no chão. \hfill (\textit{colher = objeto})
    \item int soma \hfill (\textit{inteiro})
    \item float soma \hfill (\textit{número real})
    \item class soma \hfill (\textit{tipo abstrato})
\end{itemize}

\section*{Paralelo entre sintaxe e semântica}

\begin{itemize}
    \item \textbf{Sintaxe:}
    \begin{itemize}
        \item Reconhecida antes da semântica.
        \item Estrutura matemática bem definida (Gramáticas de Chomsky).
        \item \textit{Livre} — manipula símbolos sem significados.
    \end{itemize}
    
    \item \textbf{Semântica:}
    \begin{itemize}
        \item Analisada após a sintática.
        \item Exige interpretações (mais subjetiva).
        \item \textit{Associada} — há um significado atrelado.
    \end{itemize}
\end{itemize}

\section*{Vocabulário inicial da disciplina}

\begin{itemize}
    \item \textbf{Programa sintaticamente errado:}
    \begin{itemize}
        \item O texto escrito \textit{não é aceito pela linguagem}.
        \item Exemplo: erro de compilação.
    \end{itemize}
    
    \item \textbf{Programa sintaticamente correto:}
    \begin{itemize}
        \item O texto é \textit{aceito pela linguagem}.
        \item Pode não fazer o que o programador queria (bug).
        \item É \textit{sintaticamente válido}, mas possivelmente semanticamente incorreto.
    \end{itemize}
\end{itemize}

\end{document}