\documentclass[12pt]{article}
\usepackage[utf8]{inputenc}
\usepackage[brazil]{babel}
\usepackage{amsmath, amssymb}
\usepackage{listings}
\usepackage{graphicx}
\usepackage{hyperref}
\usepackage{enumitem}
\usepackage{geometry}
\geometry{a4paper, margin=2.5cm}

\title{Anotações — Linguagens Formais e Autômatos}
\author{Disciplina: Linguagens Formais e Autômatos\\Professores: José Rui e Thiago Augusto de Oliveira}
\date{Aula 01 — 26 de Fevereiro de 2025}

\begin{document}

\maketitle

\section*{Sintaxe e Semântica}

\subsection*{O que é uma linguagem formal?}
\begin{itemize}
  \item Desenvolvida na década de 1950.
  \item Relacionada à linguagem natural e linguagens artificiais como C, Pascal, etc.
  \item Usada principalmente para análise léxica e análise sintática.
  \item Exemplos: linguagens de programação, linguagens naturais, expressões matemáticas, circuitos digitais.
\end{itemize}

\subsection*{Diferença entre sintaxe e semântica}
\textbf{Sintaxe}: disposição lógica e gramatical dos símbolos. \\
\textbf{Semântica}: significado das expressões após análise sintática.

\subsection*{Exemplos de sintaxe}
\begin{itemize}
  \item Corretos: \textit{José bebeu água. A = 45;}
  \item Incorretos: \textit{Água bebeu José. 45 = A;}
\end{itemize}

\subsection*{Exemplo de ambiguidade semântica}
\begin{itemize}
  \item Caminho (verbo) / Caminho (substantivo)
  \item Colher (verbo) / Colher (objeto)
  \item Soma (inteiro / float / classe)
\end{itemize}

\section*{Conceitos Básicos}

\subsection*{Alfabeto, palavras e concatenação}

\begin{itemize}
  \item Um alfabeto $\Sigma$ é um conjunto finito de símbolos.
  \item Uma palavra $w$ é uma sequência de símbolos de $\Sigma$.
  \item Exemplo: $\Sigma = \{a,b\}$, $w = abcb \Rightarrow |w| = 4$
  \item Operações sobre palavras: concatenação, prefixo, sufixo, subpalavra.
\end{itemize}

\section*{Gramática Formal}

\subsection*{Definição}
Uma gramática $G$ é uma tupla $G = (V, T, P, S)$, com:
\begin{itemize}
  \item $V$: variáveis (não-terminais)
  \item $T$: terminais
  \item $P$: regras de produção
  \item $S$: símbolo inicial
\end{itemize}

\subsection*{Exemplo de gramática para números (243)}

\begin{align*}
  N &\rightarrow D \\
  N &\rightarrow DN \\
  D &\rightarrow 0|1|2|3|4|5|6|7|8|9
\end{align*}

Derivação para $243$:

\[
N \Rightarrow DN \Rightarrow 2N \Rightarrow 2DN \Rightarrow 24N \Rightarrow 24D \Rightarrow 243
\]

\end{document}